%%%%%%%%%%%%%%%%%%%%%%%%%%%%%%%%%%%%%%%%%
% Medium Length Professional CV
% LaTeX Template
% Version 2.0 (8/5/13)
%
% This template has been downloaded from:
% http://www.LaTeXTemplates.com
%
% Original author:
% Trey Hunner (http://www.treyhunner.com/)
%
% Important note:
% This template requires the resume.cls file to be in the same directory as the
% .tex file. The resume.cls file provides the resume style used for structuring the
% document.
%
%%%%%%%%%%%%%%%%%%%%%%%%%%%%%%%%%%%%%%%%%

%----------------------------------------------------------------------------------------
%	PACKAGES AND OTHER DOCUMENT CONFIGURATIONS
%----------------------------------------------------------------------------------------

\documentclass{resume} % Use the custom resume.cls style

\usepackage[left=0.75in,top=0.6in,right=0.75in,bottom=0.6in]{geometry} % Document margins
\usepackage{graphicx}
\name{Fei Xia} % Your name
\address{2\# Zijing Student Apartment \\ Tsinghua University \\ Beijing 100084, P.R. China} % Your address
\address{86-15652799536(mobile) \\ \href{mailto:xf12@mails.tsinghua.edu.cn}{xf12@mails.tsinghua.edu.cn} \\ \href{http://fxia.me}{http://fxia.me}} % Your phone number and email

\begin{document}

%----------------------------------------------------------------------------------------
%	EDUCATION SECTION
%----------------------------------------------------------------------------------------

\begin{rSection}{Education}

{\bf Tsinghua University}, Beijing, China \hfill {\em 2012 - 2016(Expected)} \\ 
B.E.({\it Expected}) Department of Automation \\
GPA: 94.1/100, Average of Math Courses: 96.0/100		\\
Ranking 1\textsuperscript{st}/144 in Dept. of Automation \\ \\
{\bf Georgia Institute of Technology}, Atlanta, GA, USA \hfill {\em 2014.8 - 2014.12} \\ 
Exchange student in School of Electrical and Computer Engineering\\
GPA: 4.0/4.0 \\ \\ 
{\bf Stanford University}, Stanford, CA, USA \hfill {\em 2015.7 - 2015.9} \\ 
Undergraduate visiting research assistant in Department of Electrical Engineering \\
The Chinese Undergraduate Visiting Research (UGVR) Program, only 18 students selected from Mainland China and Taiwan\\ \\
{\bf Courses related to my research interests:} \\ 
{\bf Tsinghua}: Calculus A1: (95/100), Calculus A2: (99/100), Linear Algebra 1: (93/100), Linear Algebra 2: (95/100), Physics B1: (100/100), Physics B: (100/100), Probability and Statistics: (100/100); \\Data Structure and Algorithms: (98/100), C++ Program Design and Training: (98/100),  Interdisciplinary Research Training(in Bioinformatics): (92/100). \\
{\bf Georgia Institute of Technology}: Stochastic Processes(graduate level): (4.0/4.0), Signals and System Analysis: (4.0/4.0), Digital Signal Processing: (4.0/4.0), Computer Vision: (4.0/4.0).

\end{rSection}

%----------------------------------------------------------------------------------------
%	WORK EXPERIENCE SECTION
%----------------------------------------------------------------------------------------

\begin{rSection}{Research Experiences}

\begin{rSubsection}{Stanford University}{Stanford, CA, USA}{2015.7 - Present}{Information Systems Laboratory, Department of Electrical Engineering}{Research Assistant, Advisor: {\bf Prof. David Tse}}
\setlength\itemsep{-0.6em}
\item[] {\bf Project 1}: {\bf {\it De novo} DNA Sequence Assembly from Barcoded Reads}
\item Established information theoretic bounds for a third generation sequence technology, 10X. 
\item Designed algorithms to take advantage of barcoded linked reads to generate better assembly than state of the art. 
\item Experimented on Human Chromosome 21, and boosted N50 of state-of-the-art assembler by 30\%.
\item[] {\bf Ongoing project: A {\it de novo} Sequence Assembler for PacBio Reads Based on Sparse String Graph}
\item Able to generate {\bf finished} assembly at accuracy 99.9\% for {\it E.Coli} based on sparse string graph methods.
\end{rSubsection}


\begin{rSubsection}{Georgia Insitute of Technology}{Atlanta, GA, USA}{2014.8 - 2014.12}{Sun Lab, School of Computational Science \& Engineering,  College of Computing}{Research Assistant, Advisor: {\bf Prof. Jimeng Sun}}
\setlength\itemsep{-.6em}
\item[] {\bf Project 1:} {\bf Epilepsy Seizure Prediction Based on EEG Data}
%\item Generally applied machine learning and data mining techniques to solve healthcare problems
\item Built an analytic model for epilepsy seizure prediction based on EEG data
\item Participated in Kaggle Competition, achieved AUC 0.7298, and ranked top 8\% (out of 504 teams)   
\item[] {\bf Project 2:} {Cost Estimation for Cloud-Based Analytic Machine Learning Pipeline }
\item Conducted experiments to do estimation for running time and cost of cloud-based analytical pipeline. 
\end{rSubsection}

\begin{rSubsection}{Tsinghua University}{Beijing, China}{2014.2 - 2015.2}{Knowledge Engineering Group, Department of Computer Science and Technology}{Research Assistant, Advisor: {\bf Prof. Jie Tang}}
\setlength\itemsep{-.6em}
\item[] {\bf Project: Continuous Time Information Network Mining for Diffusion Cascades}
\item Designed models that considers indirect influence and structural influence for continuous-time information diffusion in networks 
\item Proposed gradient descent methods for learning models and making inferences. 
\item Experimented on Sina Weibo dataset and increased AUC by 10\% compared with baseline algorithm. 
\end{rSubsection}



\begin{rSubsection}{Tsinghua University}{Beijing, China}{2013.7 - 2014.7}{MOE Key Laboratory of Bioinformatics and Bioinformatics Division}{Research Program Member, Advisor: {\bf Prof. Xiaowo Wang, Prof. Zhen Xie}}
\setlength\itemsep{-.6em}
    \item[] {\bf Project:  Marvelous TALE --- Towards Better DNA Editing Tools}
	\item Developed a DNA optimizing algorithm based on genetic algorithm and multi-sequence alignment for reducing homologous recombination probability of TALE expression in {\it E.Coli}
	\item Implemented the algorithm, conducted experiments and provided data for wet-lab synthesis
	\item Participated in International Genetically Engineered Machine Competition(IGEM) 2014 and won Bronze Prize
\end{rSubsection}

%\newpage

%\begin{rSubsection}{Tsinghua University}{Beijing, China}{2013.1 - 2014.7}{Tinker@Home Group, Texas Instruments-Tsinghua Future Robots Club}{Team Leader}

%\item Worked on developing a robust face detecting and recognizing system for robots.
%\item Developed a package for face detection, alignment, archive building, and recognition.
%\item Participated in RoboCup 2014 in Jo$\tilde{a}$o Pessoa, Brazil, won 10\textsuperscript{th} place in @Home League.	
%\end{rSubsection}

%\begin{rSubsection}{Tsinghua University}{Beijing, China}{2013.9 - 2014.4}{Center for Intelligent and Networked Systems}{Research Assistant}
%\item {\bf ROSLink - A Lightweight Middleware for IoT Based on Robot Operating System}
%\item Supervisor: {\bf Prof. Qianchuan Zhao}
%\item Developed a middleware for IoT based on Robot Operating System
%\item This project won the 3\textsuperscript{rd} Prize in the 32\textsuperscript{nd} Challenge Cup in Tsinghua University

%\end{rSubsection}



%------------------------------------------------

\end{rSection}

%----------------------------------------------------------------------------------------
%	TECHNICAL STRENGTHS SECTION
%----------------------------------------------------------------------------------------
\begin{rSection}{Publications and Manuscripts}
\begin{enumerate}[label={[}\arabic*{]}]
\setlength\itemsep{-0.5em}
\item[] {\bf Conference Papers}
	\item {\bf Fei Xia}, {\it et al}. Human-aware mobile robot exploration and motion planner. {\it Proceeding of IEEE SoutheastCon 2015.}	
\item[] {\bf Manuscripts}
	\item Hang Su, {\bf Fei Xia}, Jimeng Sun, {\it et al}. Tell Me the Price First: Cost Estimation for Cloud-Based Healthcare Predictive Modeling. {\it to be submitted to Journal of Medical Internet Research}
	\item {\bf Fei Xia}, Yu Xia, Jie Tang. Continuous Time Information Network Mining for Diffusion Cascades. 	
\end{enumerate}

\end{rSection}


\begin{rSection}{Awards}

\begin{tabular}{ @{} >{\bfseries}l @{\hspace{3ex}} l }
2015 & Chang Jiong Scholarship (Highest honor in Dept. of Automation, 1 out of 560) \\

2014 & Fang Chongzhi Scholarship (Highest honor in Dept. of Automation, 1 out of 560) \\
2014 & China Scholarship Council Excellent Undergraduate Fellowship \\
2014 &Tsinghua Sparks Program (Undergraduate High-tech Club) Membership \\

2013 & National Southwest Associateo University Scholarship (1 out of 560) \\
2012 & Tsinghua University Outstanding Freshman Scholarship \\
2011 & Gold Medal of 25\textsuperscript{th} Chinese Chemical Society National Chemistry Contest (Ranking 8\textsuperscript{th}/92k)
\end{tabular}

\end{rSection}





\begin{rSection}{Technical Strengths}

\begin{tabular}{ @{} >{\bfseries}l @{\hspace{6ex}} l }
Programming Languages & Proficient in C/C++, Python, Matlab, Java \\
Tools & ROS, vim, git, cmake, gcc, \LaTeX, bash, MPI, OpenMP \\ 
Research & Familiar with state-of-the-art machine learning techniques, \\ & familiar with Next Generation Sequencing data analytics. 
\end{tabular}

\end{rSection}



\begin{rSection}{Language Skills}

\begin{tabular}{ @{} >{\bfseries}l @{\hspace{6ex}} l }


English & Excellent listening, speaking, reading and writing abilities \\ 
 & ~$\LargerCdot$ TOEFL iBT 109/120  (Reading 30, Listening 29, Speaking 24, Writing 26) \\
 & ~$\LargerCdot$ GRE Verbal 155/170, Quantitive 170/170, Analytical Writing 4.0/6.0 \\
 

\end{tabular}

\end{rSection}



%----------------------------------------------------------------------------------------
%	EXAMPLE SECTION
%----------------------------------------------------------------------------------------

%\begin{rSection}{Section Name}

%Section content\ldots

%\end{rSection}

%----------------------------------------------------------------------------------------

\end{document}
