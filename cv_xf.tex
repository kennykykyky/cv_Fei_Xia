%%%%%%%%%%%%%%%%%%%%%%%%%%%%%%%%%%%%%%%%%
% Medium Length Professional CV
% LaTeX Template
% Version 2.0 (8/5/13)
%
% This template has been downloaded from:
% http://www.LaTeXTemplates.com
%
% Original author:
% Trey Hunner (http://www.treyhunner.com/)
%
% Important note:
% This template requires the resume.cls file to be in the same directory as the
% .tex file. The resume.cls file provides the resume style used for structuring the
% document.
%
%%%%%%%%%%%%%%%%%%%%%%%%%%%%%%%%%%%%%%%%%

%----------------------------------------------------------------------------------------
%	PACKAGES AND OTHER DOCUMENT CONFIGURATIONS
%----------------------------------------------------------------------------------------

\documentclass{resume} % Use the custom resume.cls style

\usepackage[left=0.75in,top=0.6in,right=0.75in,bottom=0.6in]{geometry} % Document margins
\usepackage{graphicx}
\name{Fei Xia} % Your name
\address{2\# Zijing Student Apartment \\ Tsinghua University \\ Beijing 100084, P.R. China} % Your address
\address{86-15652799536(mobile) \\ xf1280@gmail.com} % Your phone number and email

\begin{document}

%----------------------------------------------------------------------------------------
%	EDUCATION SECTION
%----------------------------------------------------------------------------------------

\begin{rSection}{Education}

{\bf Tsinghua University}, Beijing, China \hfill {\em 2012 - 2016(expected)} \\ 
B.S. Department of Automation \\
GPA: 95.3/100\\
Ranking 1\textsuperscript{st}/144 in Dept. of Automation

{\bf Georgia Institute of Technology}, Atlanta, GA, USA \hfill {\em 2014.8 - 2014.12} \\ 
Exchange student in School of Electrical and Computer Engineering\\
GPA: 4.0/4.0


\end{rSection}

%----------------------------------------------------------------------------------------
%	WORK EXPERIENCE SECTION
%----------------------------------------------------------------------------------------

\begin{rSection}{Research Experiences}

\begin{rSubsection}{Georgia Insitute of Technology}{Atlanta, GA, USA}{2014.8 - Present}{Sun Lab, School of Computational Science \& Engineering,  College of Computing}{Research Assistant}
\item  {\bf Epilepsy Seizure Prediction Based on EEG Data}
\item Supervisor: Prof. Jimeng Sun
\item Generally applied machine learning and data mining techniques to solve healthcare problems
\item Built an analytic model for epilepsy seizure prediction based on EEG data
\item Participated in Kaggle Competition, and ranked top 10\%.   
\item {\bf Ongoing project: Cost Estimation for Cloud-Based Analytic Machine Learning Pipeline }
\end{rSubsection}



\begin{rSubsection}{Tsinghua University}{Beijing, China}{2014.2 - Present}{Knowledge Engineering Group, Department of Computer Science and Technology}{Research Assistant}
\item {\bf Influence Propagation Modeling based on Factor Graph Model and Hawkes Process}
\item Supervisor: Prof. Jie Tang
\item Designed various models for continuous-time information diffusion in networks. 
\item Proposed methods for learning models and making inferences. 
\end{rSubsection}



\begin{rSubsection}{Tsinghua University}{Beijing, China}{2013.7 - 2014.7}{Center for Synthetic and System Biology\\ MOE Key Laboratory of Bioinformatics and Bioinformatics Division}{Research Program Member}
    \item {\bf Marvelous TALE --- Towards Better DNA Editing Tools}
	\item Supervisor: Prof. Xiaowo Wang, Prof. Zhen Xie
	\item Developed a DNA optimizing algorithm for reducing homologous recombination probability of TALE expression in E.Coli
	\item Implemented the algorithm, conducted experiments and provided data for wet-lab synthesis
	\item Participated in International Genetically Engineered Machine Competition 2014 and won Bronze Prize
\end{rSubsection}

\newpage

\begin{rSubsection}{Tsinghua University}{Beijing, China}{2013.1 - 2014.7}{Tinker@Home Group, Texas Instruments-Tsinghua Future Robots Club}{Team Leader}

\item Worked on developing a robust face detecting and recognizing system for robots.
\item Developed a package for face detection, alignment, archive building, and recognition.
\item Participated in RoboCup 2014 in Jo$\tilde{a}$o Pessoa, Brazil, won 10\textsuperscript{th} place in @Home League.	
\end{rSubsection}

\begin{rSubsection}{Tsinghua University}{Beijing, China}{2013.9 - 2014.4}{Center for Intelligent and Networked Systems}{Research Assistant}
\item {\bf ROSLink - A Lightweight Middleware for IoT based on Robot Operating System}
\item Supervisor: Prof. Qianchuan Zhao
\item Developed a middleware for IoT based on Robot Operating System
\item This project won the 3\textsuperscript{rd} Prize in the 32\textsuperscript{nd} Challenge Cup in Tsinghua University

\end{rSubsection}



%------------------------------------------------

\end{rSection}

%----------------------------------------------------------------------------------------
%	TECHNICAL STRENGTHS SECTION
%----------------------------------------------------------------------------------------


\begin{rSection}{Awards}

\begin{tabular}{ @{} >{\bfseries}l @{\hspace{3ex}} l }
2014 & Chongzhi Fang Scholarship (Highest honor in Dept. of Automation, 1 out of 560) \\
2014 & China Scholarship Council Excellent Undergraduate Fellowship \\
2014 &Tsinghua Sparks Program (Undergraduate High-tech Club) Membership\\

2013 & National Southwest Associateo University Scholarship(1 out of 560) \\
2012 & Tsinghua University Outstanding Freshman Scholarship \\
2011 & Gold Medal of 25\textsuperscript{th} Chinese Chemical Society National Chemistry Contest (Ranking 8\textsuperscript{th}/92k)
\end{tabular}

\end{rSection}





\begin{rSection}{Technical Strengths}

\begin{tabular}{ @{} >{\bfseries}l @{\hspace{6ex}} l }
Programming Languages & Proficient in C/C++, Python, Matlab, Java \\

Tools & ROS, vim, git, cmake, gcc, \LaTeX \\ 

Research & Knowledgeable with state-of-the-art machine learning techniques 
\end{tabular}

\end{rSection}



\begin{rSection}{Language Skills}

\begin{tabular}{ @{} >{\bfseries}l @{\hspace{6ex}} l }


English & Excellent listening, speaking, reading and writing abilities \\ 
 & ~$\cdot$ TOEFL iBT 111/120 \\
Chinese & Mother tongue \\

\end{tabular}

\end{rSection}



%----------------------------------------------------------------------------------------
%	EXAMPLE SECTION
%----------------------------------------------------------------------------------------

%\begin{rSection}{Section Name}

%Section content\ldots

%\end{rSection}

%----------------------------------------------------------------------------------------

\end{document}
